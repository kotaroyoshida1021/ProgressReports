\section{入力をなぜGPで表現したか}
			タイトルの件について,おそらく研究会の説明等では不足であると考え,追記します.
			
			この表現の目的はある関数$ f(x) $の特徴を記述するパラメータを入力パラメータとしたいと考えたところから出発します.最も単純で馴染みの深い方法としては,下記のように$ f(x) $を基底関数$ \bd{\phi} = \{ \phi_i(x) \} $の重み付き線形和によって表現する方法です.
			\begin{equation}\label{eq:LinearEq}
				f(x) = \sum_{i=0}^{N} w_i \phi_i(x)  = \bd{\phi} \bd{w}
			\end{equation}
			入力のデータセット$ \bd{x} = \{x_i\} $出力データセット$ \bd{\hat{f}} = \{\hat{f}_i\}$が与えられている場合,$ \bd{w} $は次のような二乗誤差を最小化することで得られる.
			\begin{equation}\label{eq:ErrorToMinimize}
				E =  \sum_{i=0}^{N} (\hat{f}_i - \sum_{j=0}^{N} w_j \phi_j(x_i))^2
			\end{equation}
			しかしこの場合,$ \dim(\bd{w}) $をどの程度であればよいのかは関数毎に設定が必要である点や,パラメータの次元が増加してしまう,などの問題があります.
			
			ガウス過程では,この$ \bd{w} $があるガウス分布$ \mathcal{N}(\bd{0},\lambda^2 \bd{I}) $から生成されたものと仮定します.この場合,出力の$ f $もまたガウス分布に従います.平均と分散はそれぞれ次式のように計算されます.
			\begin{equation}\label{eq:Avg}
				\bd{m} = \mathbb{E}(f(x)) = \bd{\phi} \cdot \mathbb{E}( \bd{w}) = \bd{0}
			\end{equation}
			\begin{equation}\label{eq:Var}
				\bd{\sigma} = \mathbb{E}(\bd{f} - \bd{m})^2 = \lambda^2  \bd{\Phi}  \bd{\Phi}^{\mathrm{T}}
			\end{equation}
			ただし,$ \bd{\Phi} =\{\phi_j(x_i) \}$とおいております.これを$ \bd{K} = \lambda^2  \bd{\Phi}  \bd{\Phi}^{\mathrm{T}} $とおくと$ \bd{K} $は$ f $の共分散行列となり,$ f $はガウス分布$ \mathcal{N}(\bd{0},\bd{K}) $に従い,$ \bd{K} $の各成分は$ k_{i,j} = \bd{\phi}(x_i) \bd{\phi}^{\mathrm{T}}(x_j) $となります.この時$ f $のガウス分布を求めるためには,$ \bd{\phi} $を設計するのではなく,$ k_{i,j} $を直接設計すればよいことが分かります.このようなテクニックはカーネルトリックと呼ばれ,$ k_{i,j} $をカーネル関数と呼びます.
			
			重要な点は2点で,一つは$ \dim(\bd{\phi}) $の次元が無限としても,$ k_{i,j} $が有限の値を持つようであれば構わないという点で,もう一つの点は,ガウス過程の出発が基底関数の線形和で表すというRitz法と考え方が等価である点です.
			
			したがって,カーネル関数を決定するいくつかのパラメータを適切に設定さえすれば,$ f $をガウス分布によって表現でき,これにより,パラメータの次元増加を避けることができると考えられます.カーネル関数のパラメータを求める際には,対数尤度最適の考え方が用いられ,確率分布の対数をとった関数が最大化するようにパラメータを最適化します.
			\begin{equation}\label{eq:Suudo}
			\log(p) = -\log|\bd{K}_{\theta}| - \bd{y}^T \bd{K}_{\theta}^{-1} \bd{y}+C
			\end{equation}