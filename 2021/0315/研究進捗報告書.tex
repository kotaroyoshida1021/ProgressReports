\documentclass[16pt]{jsarticle}

\usepackage{SPR}

\headerSPR
\begin{document}
	\titleSPR{\number\year}{\number\month}{\number\day}{D2}{吉田 皓太郎}
%%%%%%%%%%%%%%%%%%%%%%%%%%%%%%%%%%%%%%
	%\articleSPRabst
	%	\begin{itemize}
	%		\item =どういう事態か
	%	\end{itemize}
		
%今回のMTGの目的,何をディスカッションし,結果に何を得たいのかを決定する.※ミーティングごとに記入
	\articleSPRobj
		今回のMTGでは,D論の構成を現状までに達成された項目に基づき考察・再構築したものを記載しており,先生の意見を踏まえD論構成を確定させたいと考えている.また,そのD論構成に基づいた時,残りの少ない時間で何を行うべきかを議論し,今後の計画へと繋げたいと考えている.
		

%%%%%%%%%%%%%%%%%%%%%%%%%%%%%%%%%%%%%%
% 1.前回からのノルマ
	\articleSPRitemsone
		%\begin{enumerate}
		%	\item A
		%\end{enumerate}
		
		\tableofcontents
		
		
%%%%%%%%%%%%%%%%%%%%%%%%%%%%%%%%%%%%%%
%\begin{itemize}
%	\item 新規手法について
%	\item ISFAアウトライン
%\end{itemize}
%%%%%%%%%%%%%%%%%%%%%%%%%%%%%%%%%%%%%%
% 2.具体的な成果
	\articleSPRitemstwo
	\renewcommand{\labelitemi}{$\blacktriangledown$}
	%\renewcommand{\labelitemi}{$\bigcirc$}
	\newcommand{\argmax}{\mathop{\rm arg~max}\limits}
	\newcommand{\argmin}{\mathop{\rm arg~min}\limits}
	\newcommand{\Ker}{{\rm Ker}}
	\newcommand{\rank}{{\rm rank}}
%%%%%%%%%%%%%%%%%%%%%%%%%%%%%%%%%%%%%
	\section{【関連】現状で達成されていることのまとめ}
		D論構成を考えるにあたって,現状まででどのような研究成果があるかについてまとめておきます.また,論文ベースでの状況は別紙「【関連】論文まとめ」に記載します.
		\begin{itemize}
			\item 微分幾何学をベースに,ある2つの三次元曲線間に形成される可展面およびその展開形状の導出
				\begin{enumerate}
					\item 測地的曲率を最適化する手法(B4)
					\item 2つの弧長間の関係を求める手法(M1)
				\end{enumerate}
			\item ある形状制約条件を満たす可展面形状の導出-局所変形を抑えるという意味でポテンシャルエネルギー(力学的立場だと厳しいため,あくまで曲げ量の総量という意味で)を最小化
			\item ある点群に合わせた可展面の導出(ブラジャーカップを例に,2枚接ぎカップで)
			\item 形状に対する評価値を機械学習によって予測する出力予測器の開発(出力から入力形状予測も考慮できるようにしている特徴)
			\item 【予備項目】百崎でやったような詳細設計に向けたもの(D論載せる可能性0ではない)
			\item 【予備項目】微小変形を仮定した可展面変形形状の導出(ボツネタ臭い)
		\end{itemize}
	\section{D論構成}
		\subsection*{【大枠の構成まとめ】}
		現状のD論構成について,現状の考察を行った結果をまとめました.	D論の最終目標は,「設計要求に合わせた可展面をCADなどで利用可能なサーフェスモデルとして出力する」ことを設定する.この時,目標のために必要な要素は2つにまとめられます.()には,荒井先生が過去に書いて下さったD論に関する意見の中で使われていた言葉を,自分なりに当てはめてみました.
		\begin{itemize}
			\item 可展面を表現する形状モデル(理論部分)
			\item 形状をモデルを利用し,設計要求に合わせて形状を導出する方法(応用部分)
		\end{itemize}
		これを踏まえて,D論の大枠は以下のような構成にまとめることができる.
		\begin{enumerate}
			\item 緒言
			\item 形状モデルに関する提案
			\item 設計要求に基づく設計支援手法
			\item ケース
			\item 結言
		\end{enumerate}
		章立てに関しては,上記の流れでまとめ,全5章立てにするとよいのではないかと思われる.やったことに対して細かく章分けしてもいいかもしれないが,ケーススタディを最後にまとめる場合には文量が薄くなってしまうことが予想される.
		
		形状モデルおよび設計要求に基づく設計支援手法には,以下のような概要から話を広げていくといいかと思われる.
		\begin{itemize}
			\item 形状モデルに関しては,微分幾何学をベースにした展開で形状制約を考慮した部分までを記載
			\item 設計要求に関しては,リバースエンジニアリング,設計評価値などの話を交えながら,点群・機械学習の話を展開する.
		\end{itemize}
		\subsection*{【これからの行動選択など】}
			先ほどの構成を基にすると,形状モデルまたは設計要求のどちらを扱うか(あるいはどちらも?)を考える必要がある.
			
			\begin{itemize}
				\item 形状モデルに関しては,SCIなどで扱ってきた「修正」がキーワードであると想定される.B-splineのように局所的変形を考えるなど,どちらかといえば想定としてCADの機能にフォーカスし,力学的要素などは極力排除されるべきであると思われる.
				\item 一方,設計要求に関しては,百崎が行なっていた研究をブラッシュアップした研究になるとよい(おそらくそのままの結果を使うことはできないかなというのが正直な感想)
				\item また,設計要求に関する研究の応用として,点群は二枚接ぎでなく複数枚対応できるようにする(設計者が枚数を入力する?)ことや,多目的最適化への発展などが挙げられる(機械学習要求+と何かを最小化・最大化)
			\end{itemize}
			また,緒言について述べる際には,CADシステムの概要を加えるか否かも考える必要がある.その場合には,可展面までの話の流れが十分ロジカルである必要がある.さらに,既存の形状モデルについての歴史などをまとめる必要があり,それを緒言に記載するか,あるいは形状モデルに入れるかという部分も迷いどころである.
		\subsection*{【調査することリスト】}
			\begin{itemize}
				\item 既存の形状モデルの表現方法について(B-spline,ベジェなど)
				\item 機械学習関連の研究を立たせるための機能群調査(引き続き)
			\end{itemize}
	\section{次回のMTGについて(終了後記載)}
	\begin{itemize}
		\item  
		\item 
	\end{itemize}
	###
	\newpage
%\vspace{10cm}
%%%%%%%%%%%%%%%%%%%%%%%%%%%%%%%%%%%%%%
% 3.達成できなかったこととその問題点
	%\articleSPRthree
	
%%%%%%%%%%%%%%%%%%%%%%%%%%%%%%%%%%%%%%

%\vspace{14cm}
%%%%%%%%%%%%%%%%%%%%%%%%%%%%%%%%%%%%%%
	%\articleSPRfour
	%\articleSPRfive
\end{document}
