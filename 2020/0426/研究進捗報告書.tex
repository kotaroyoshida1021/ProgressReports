\documentclass[11pt]{jsarticle}

\usepackage{SPR}

\headerSPR
\begin{document}
	\titleSPR{\number\year}{\number\month}{\number\day}{D1}{吉田 皓太郎}
%%%%%%%%%%%%%%%%%%%%%%%%%%%%%%%%%%%%%%
	\articleSPRabst
		\begin{itemize}
			\item 定性的な機能を満たすカップの設計支援
		\end{itemize}
		
		
	\articleSPRobj
		\begin{enumerate}
			\item 定性的な要求を,最適化問題によって表現することで,数理処理が可能となる.
			\item 事例をベースとした機械学習を用いることで,ある程度の確かさが確保できると思われる.
		\end{enumerate}
%%%%%%%%%%%%%%%%%%%%%%%%%%%%%%%%%%%%%%
% 1.前回からのノルマ
	\articleSPRitemsone
		%\begin{enumerate}
		%	\item A
		%\end{enumerate}
		
		\tableofcontents
		
		
%%%%%%%%%%%%%%%%%%%%%%%%%%%%%%%%%%%%%%
%\begin{itemize}
%	\item 新規手法について
%	\item ISFAアウトライン
%\end{itemize}
%%%%%%%%%%%%%%%%%%%%%%%%%%%%%%%%%%%%%%
% 2.具体的な成果
	\articleSPRitemstwo
\renewcommand{\labelitemi}{$\blacktriangledown$}
%%%%%%%%%%%%%%%%%%%%%%%%%%%%%%%%%%%%%%
	\section{研究進捗}
		\subsection{ベイズ最適化を扱ってみて}
			\subsubsection{制約条件について}
				ベイズ最適化やガウス過程回帰はpythonのGPyおよびGPyOptで実行できる.ベイズ最適化は制約条件を取り扱うことは可能であるが,それは不等式の形でしか入力できない.制約条件を考慮するためには,以下の3つが考えられる.
				\begin{enumerate}
					\item 拡張ラグランジュ乗数法(今までのやつ)
					\item 近似的な不等式$ |c_i|<\varepsilon $の導入
					\item 制約式をペナルティとした拡張目的関数を作成する.例)$ \exp(\gamma|c_i|^2) $のような形で制約式に代入
				\end{enumerate}
				1のメリットは,これまで取り扱ってきた最適化問題と同じように扱うことができる点である.したがって,数字の見方としてはこれまでと同じように扱える.1のデメリットとしては,事前学習データを用いたベイズ最適化において,手法が有効であるかが定かではないという点である.
				
				2のメリットは,ライブラリに実装されており,論文としても裏付けがあるため,手法は確実に有効である点と,実装がほとんど必要ない点である.2のデメリットは,$ \varepsilon $(制約満足度)が小さいとき,非常に収束が遅くなってしまうことである.(実際にGpyOptで後述する最適化問題において$ \varepsilon=1.0e-4 $あたりで急に速度が落ちてしまう.この最適化問題が二変数であり,凸最適化であることを加味すると,実際に適用するには不安が残るかもしれない)
				
				\begin{eqnarray}
					&\min& \; x+y \\
					&\mathrm{s.t.}& \; x^2 + y^2 =1
				\end{eqnarray}
				
				3のメリットは一番単純な形で実装可能であり,これまでで最も新規性が高いと言える点である.デメリットは,係数によって収束が大きく変化することや,数学的裏付けが乏しく,最適化問題として機能しているかが微妙である点である.
				
				先生は,どれが一番よいと思われますか.
			\subsubsection{定義域について}
				ベイズ最適化を適用するにあたって,必ず定義域を指定する必要がある.この定義域をどのように設定するかという問題である.もし,事前学習のデータ付近のみではなく,もっと広い範囲を探す場合,仮にそのデータが,事前学習データと大きく異なる場合は,クラスタリングの信頼度が下がってしまう問題が考えられる.したがって,定義域を拡張したい場合は注意が必要であるかもしれない.もしも定義域を拡張するべきであるという考え方を取るならば,以下の方法を提案する.(4月19に提出した進捗報告書のものを改良する)クラスタリングの結果を出力する関数を$ \Phi(\bd{a}) $とすると,出力の微小変化量$ \Delta \Phi $は,入力パラメータの微小変化量$ \Delta \bd{a} $を用いて
				
				\begin{equation}\label{eq:DelPhi}
					\Delta \Phi = \Phi(\bd{a}+\Delta \bd{a}) - \Phi(\bd{a}) \simeq \nabla \Phi \Delta \bd{a}
				\end{equation}
				となる.もし,$ \Delta \bd{a} $の変動に対し$ \Delta \Phi $がmaxの状態=1を保ったままであるとすると,$ \nabla \Phi \Delta \bd{a} $は0である必要がある.このとき,$ \Delta \bd{a} $を以下のように定めることで\eqref{eq:DelPhi}を満たすことができる.
				\begin{equation}\label{eq:}
					\Delta \bd{a} = \eta \frac{1}{|\nabla \Phi|} \left(\frac{\partial \nabla \Phi}{\partial a_j} -\left(\frac{\partial \nabla \Phi}{\partial a_j} \cdot \bd{t}\right)\bd{t} \right)
				\end{equation}
				ただし,$ t = \nabla \Phi / |\nabla \Phi| $であり,$ \eta $は小さい数である.$ j $の決め方は,$ \min_j |\Delta \bd{a}| $と定める.以上から,maxの状態を保ちつつ,変数の定義域を拡張することができると思われる.
		\section{ワコールミーティングのその後について}
			その後音沙汰ないですか?いずれにしてもカップの変形予測は取り扱う予定なのですが,最終的なゴールを「カップの変位量と型紙形状との相関を応答曲面法を用いて,回帰式を作成する.」にしようかと思っております.理由としては,布物性はどちらかというと最適化するものではなく,与えるものである(設計段階,あるいはもっと上流の企画の段階から決まっている可能性がある)ためである.そのため,どちらかというと布に合わせてパターンが変わるという考え方の方が自然であるかと思われます.
		
		
	\newpage
\vspace{10cm}
%%%%%%%%%%%%%%%%%%%%%%%%%%%%%%%%%%%%%%
% 3.達成できなかったこととその問題点
	%\articleSPRthree
%%%%%%%%%%%%%%%%%%%%%%%%%%%%%%%%%%%%%%

\vspace{14cm}
%%%%%%%%%%%%%%%%%%%%%%%%%%%%%%%%%%%%%%
	\articleSPRfour
	\articleSPRfive
\end{document}
