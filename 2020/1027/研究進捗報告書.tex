\documentclass[11pt]{jsarticle}

\usepackage{SPR}

\headerSPR
\begin{document}
	\titleSPR{\number\year}{\number\month}{\number\day}{D2}{吉田 皓太郎}
%%%%%%%%%%%%%%%%%%%%%%%%%%%%%%%%%%%%%%
	\articleSPRabst
		\begin{itemize}
			\item 機械学習を用いたカップ形状の設計支援
			\item 着後形状予測のためのカップの変形解析
		\end{itemize}
		
		
	\articleSPRobj
		\begin{enumerate}
			\item 定性的な機能要求を満たせるようなカップ形状を設計できる
			\item 布の物性とカップのパターンがどのような結びつきを持っているかを調べることができる.
		\end{enumerate}
%%%%%%%%%%%%%%%%%%%%%%%%%%%%%%%%%%%%%%
% 1.前回からのノルマ
	\articleSPRitemsone
		%\begin{enumerate}
		%	\item A
		%\end{enumerate}
		
		\tableofcontents
		
		
%%%%%%%%%%%%%%%%%%%%%%%%%%%%%%%%%%%%%%
%\begin{itemize}
%	\item 新規手法について
%	\item ISFAアウトライン
%\end{itemize}
%%%%%%%%%%%%%%%%%%%%%%%%%%%%%%%%%%%%%%
% 2.具体的な成果
	\articleSPRitemstwo
	\renewcommand{\labelitemi}{$\blacktriangledown$}
	%\renewcommand{\labelitemi}{$\bigcirc$}
	\newcommand{\argmax}{\mathop{\rm arg~max}\limits}
	\newcommand{\argmin}{\mathop{\rm arg~min}\limits}
%%%%%%%%%%%%%%%%%%%%%%%%%%%%%%%%%%%%%
	\section{研究進捗について}
		\begin{itemize}
			\item プログラム実装の件,最初のパラメータを求める際における,学習時に使用したグラム行列の逆行列が求められないという問題が発生中しました.この課題は,仮想的に小さな値$ \varepsilon $を持たせた単位行列を足し合わせ,これの逆行列をとることで,問題を回避しました.
			\item ISIGHTを用いて最適化計算を行なわせようと試みましたが,ExcelのVersionが結局合わないようで,頓挫しました.そのため,別の最適化手法を用いて行う予定です.(今回の計算では,次元が大幅に増加しているため,シンプルなNelder-meadでは対応しきれないため)
			
			参考)Nelder-Mead法は,50次元を超えるあたりで計算が不安定になることが知られている.
			\item ISIGHTを用いて計算を行う際は,$ \omega_{\eta} $および$ D(s) $に以下の不等式があり,これを常に満たすようにする.
			\begin{equation}\label{eq:OmgEtaEq}
				|\omega_{\eta}|<|\kappa|
			\end{equation}
			\begin{equation}
				0<D<\left| \frac{\cos \alpha}{\alpha'+\omega_{\eta}} \right|
			\end{equation}
		\end{itemize}
	\newpage
\vspace{10cm}
%%%%%%%%%%%%%%%%%%%%%%%%%%%%%%%%%%%%%%
% 3.達成できなかったこととその問題点
	%\articleSPRthree
	
%%%%%%%%%%%%%%%%%%%%%%%%%%%%%%%%%%%%%%

\vspace{14cm}
%%%%%%%%%%%%%%%%%%%%%%%%%%%%%%%%%%%%%%
	\articleSPRfour
	\articleSPRfive
\end{document}
