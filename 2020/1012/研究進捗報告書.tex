\documentclass[11pt]{jsarticle}

\usepackage{SPR}

\headerSPR
\begin{document}
	\titleSPR{\number\year}{\number\month}{\number\day}{D2}{吉田 皓太郎}
%%%%%%%%%%%%%%%%%%%%%%%%%%%%%%%%%%%%%%
	\articleSPRabst
		\begin{itemize}
			\item 機械学習を用いたカップ形状の設計支援
			\item 着後形状予測のためのカップの変形解析
		\end{itemize}
		
		
	\articleSPRobj
		\begin{enumerate}
			\item 定性的な機能要求を満たせるようなカップ形状を設計できる
			\item 布の物性とカップのパターンがどのような結びつきを持っているかを調べることができる.
		\end{enumerate}
%%%%%%%%%%%%%%%%%%%%%%%%%%%%%%%%%%%%%%
% 1.前回からのノルマ
	\articleSPRitemsone
		%\begin{enumerate}
		%	\item A
		%\end{enumerate}
		
		\tableofcontents
		
		
%%%%%%%%%%%%%%%%%%%%%%%%%%%%%%%%%%%%%%
%\begin{itemize}
%	\item 新規手法について
%	\item ISFAアウトライン
%\end{itemize}
%%%%%%%%%%%%%%%%%%%%%%%%%%%%%%%%%%%%%%
% 2.具体的な成果
	\articleSPRitemstwo
	\renewcommand{\labelitemi}{$\blacktriangledown$}
	%\renewcommand{\labelitemi}{$\bigcirc$}
	\newcommand{\argmax}{\mathop{\rm arg~max}\limits}
	\newcommand{\argmin}{\mathop{\rm arg~min}\limits}
%%%%%%%%%%%%%%%%%%%%%%%%%%%%%%%%%%%%%
	\section{研究進捗について}
		\begin{itemize}
			\item ねんどはとりあえず買いましたので,こちらでとりあえず作ってみます.
			\item プログラム実装の件,最初のパラメータを求める際における,学習時に使用したグラム行列の逆行列が求められないという問題が発生中しました.
			この問題は,
			\begin{itemize}
				\item カーネル行列の計算の仕方が間違っている
				\item モデルの精度がよくない
			\end{itemize}
		
			という課題に回帰できます.1番目はすでに取り組んだものの,解決には至らなかったため,現在2番目のところを取り組んでおります.
			\item 先週の百崎のMTGで出た案を応用し,曲面の修正作業の最適化という分野につなげることができないか,という考察を考えました.
		\end{itemize}
	\section{次回について}
		次回(来週)なのですが,できればスキップで,再来週の通常通りの時間帯(水曜15:00~)からお願いします
	\newpage
\vspace{10cm}
%%%%%%%%%%%%%%%%%%%%%%%%%%%%%%%%%%%%%%
% 3.達成できなかったこととその問題点
	%\articleSPRthree
	
%%%%%%%%%%%%%%%%%%%%%%%%%%%%%%%%%%%%%%

\vspace{14cm}
%%%%%%%%%%%%%%%%%%%%%%%%%%%%%%%%%%%%%%
	\articleSPRfour
	\articleSPRfive
\end{document}
