\section{ガウス過程について}
ある入力パラメータ$ \bd{x} $と,出力パラメータ$ \bd{y} $が,正規分布$ N(\bd{\mu},\bd{\Sigma}) $に従うとき,出力がガウス過程に従うという.このガウス過程は,ニューラルネットワークの中間ノード数無限の極限としても知られる(Neal,1996).
	\subsection{ガウス過程の概要}
		入力パラメータ$ \bd{x} $と,出力パラメータ$ \bd{y} $について,その関係$ \bd{y} = \bd{f}(\bd{x})$を求めることを考える.単純な手法としては,以下のように,基底関数$ \bd{\Phi}(\bd{x}) $を用いて,以下のような重み付きの線形和で表し,この重みを最小二乗法などで求める手法である.
		\begin{equation}\label{eq:Ritz_ppoi}
			\bd{y} = \bd{\Phi}\bd{w}
		\end{equation}
		この基底関数には,以下のような動径基底関数を用いればよいとされ,この方法は,動径基底関数回帰と呼ばれる.
		\begin{equation}\label{eq:phieq}
			\phi_i = \{\bd{\Phi}\}_i = \exp\left(-\frac{(x-\mu_h)^2}{\sigma^2}\right)
		\end{equation}
		しかし,この手法は$ \mu_h $を,入力の定義域上になるべく多く配置させる必要があり,$ \bd{x} $の次元に対して指数的に増加する\footnote{次元の呪いとも呼ばれる}ため,現実的に解くのは非常に難しいとされる.
		
		そこで,$ \bd{w} $があるガウス分布$ N(0,\lambda^2\bd{I}) $に従うとするときについて考える.このとき,$ \bd{y} = \bd{\Phi}\bd{w} $もまた,ガウス分布に従う.このとき,モデルの期待値および分散は以下のように求められる.
		\begin{eqnarray}\label{eq:VarandAvg}
			E[\bd{y}] &=& E[\bd{\Phi}\bd{w}] = \bd{\Phi}E[\bd{w}]=0\\
			\Sigma &=& E[\bd{y}\bd{y}^T]-	E[\bd{y}]E[\bd{y}^T] = \lambda^2\bd{\Phi}\bd{\Phi}^T
		\end{eqnarray}
		よって,$ \bd{y} $はガウス分布$ N(\bd{0},\lambda^2\bd{\Phi}\bd{\Phi}^T) $に従う.$ \bd{K} =  \lambda^2\bd{\Phi}\bd{\Phi}^T$と置けば,$ \bd{K} $が求めれば,$ \bd{y} $のガウス分布を求めることができる.
		この$ \bd{K} $は,グラム行列,またはカーネル行列と呼ばれ,その行列が下記の条件を満たすように設計されなければならない.
		\begin{itemize}
			\item 逆行列の存在を保証する.
			\item 正定値を持つ.
			\item 固有値が全て正である
		\end{itemize}
		カーネル行列の各成分を示すカーネル関数として代表的なものは,以下で示すRBFがある.なお,$ \bd{\theta} = [\theta_1,\theta_2] $は,ハイパーパラメータと呼ばれる.
		\begin{equation}\label{eq:Kernel}
			k(\bd{x}_i,\bd{x}_j) = \theta_1 \exp\left(- \frac{|\bd{x}_i-\bd{x}_j|^2}{2 \theta_2^2}\right)
		\end{equation}
		このモデルを用いて,ある事前学習パラメータ$ \bd{X}=[\bd{x}_1,\bd{x}_2,\cdots,\bd{x}_n] $に対し,$ \bd{Y}=[y_1,y_2,\cdots,y_n] $という出力が得られている場合において,ある入力パラメータ$ \bd{x}_* $に対する$ y_* $は事後確率の期待値として表される.
		\begin{equation}
			y_* = \bd{k}_* \bd{K}^{-1} \bd{Y}
		\end{equation}
		ただし,$ \bd{k}_* = \{k(\bd{x}_*,\bd{x}_i)\}_{i=1}^{n} $である.
		
		ハイパーパラメータの定め方について述べる.このとき,確率関数の対数をとった尤度関数は以下のように表される.
		\begin{equation}\label{eq:Suudo}
			\log(p) = -\log|\bd{K}_{\theta}| - \bd{y}^T \bd{K}_{\theta}^{-1} \bd{y}+C
		\end{equation}
		ただし,ハイパーパラメータに無関係な定数をまとめて$ C $とおいている.一般にはこの尤度関数を最大化するような最適化問題を解くことによって,パラメータを得る.この最適化問題は解析的に微分を求めることができるため,勾配法を用いて解くケースが多い.
	\subsection{クラスタリングについて}
		出力として二値的な値を取る場合,出力の値を近似的に$ y = \sigma(\bd{x}) $と表す,ただし,$ \sigma(\bd{x}) $はシグモイド関数である.このとき,事後確率の出力は少し複雑となり,以下のように表される.
		\begin{equation}\label{eq:y_*=1}
			q = \Psi\left( \frac{\bd{k}_*^T \tilde{\bd{K}}^{-1} \tilde{\bd{\mu}}}{\sqrt{1+k(\bd{x}_*,\bd{x}_*)-\bd{k}_*^T \tilde{\bd{K}}^{-1} \bd{k}_*}} \right)
		\end{equation}
		ここで,$ \Psi $は累積分布関数を表す.また,$ \tilde{\bd{K}} $や$ \tilde{\bd{\mu}} $は特定のアルゴリズムを用いて求められる(事後学習のデータを用いることなく)が,ここでは省略する.
	\subsection{ベイズ最適化について}
	ベイズ最適化は,ガウス過程を応用する形になっており,ある目的関数$ f(\bd{x}) $を最小化するような$ \bd{x} $を求める手法である.ここでは,ある学習データが与えられる場合について考える(学習データがない場合は,一様に初期解を生成する手法があるらしい?).このとき,この最適化問題は,事後学習における平均および分散の関数で定める適当な関数の最適化問題に変換でき,関数を明示的に与える必要がないのが,ベイズ最適化の特徴である.この目的関数の定め方は色々あるが,ポピュラーなものとしてはEIがある.
	\begin{equation}\label{eq:Zeq}
		Z = \frac{\mu - f^+ -\xi}{\sigma}
	\end{equation}
	\begin{equation}
		a_{EI} = (\mu - f^+ -\xi)\Psi(Z) + \sigma \psi(Z)
	\end{equation}
	ただし,$ \psi(x) \sim N(0,1)$の正規分布関数である.