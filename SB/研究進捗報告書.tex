\documentclass[16pt]{jsarticle}

\usepackage{SPR}

\headerSPR
\begin{document}
	\titleSPR{\number\year}{\number\month}{\number\day}{D2}{吉田 皓太郎}
%%%%%%%%%%%%%%%%%%%%%%%%%%%%%%%%%%%%%%
	%\articleSPRabst
	%	\begin{itemize}
	%		\item =どういう事態か
	%	\end{itemize}
		
%今回のMTGの目的,何をディスカッションし,結果に何を得たいのかを決定する.※ミーティングごとに記入
	\articleSPRobj
		今回のMTGでは,D論の構成を現状までに達成された項目に基づき考察・再構築したものを記載しており,先生の意見を踏まえD論構成を確定させたいと考えている.また,そのD論構成に基づいた時,残りの少ない時間で何を行うべきかを議論し,今後の計画へと繋げたいと考えている.
		

%%%%%%%%%%%%%%%%%%%%%%%%%%%%%%%%%%%%%%
% 1.前回からのノルマ
	\articleSPRitemsone
		%\begin{enumerate}
		%	\item A
		%\end{enumerate}
		
		\tableofcontents
		
		
%%%%%%%%%%%%%%%%%%%%%%%%%%%%%%%%%%%%%%
%\begin{itemize}
%	\item 新規手法について
%	\item ISFAアウトライン
%\end{itemize}
%%%%%%%%%%%%%%%%%%%%%%%%%%%%%%%%%%%%%%
% 2.具体的な成果
	\articleSPRitemstwo
	\renewcommand{\labelitemi}{$\blacktriangledown$}
	%\renewcommand{\labelitemi}{$\bigcirc$}
	\newcommand{\argmax}{\mathop{\rm arg~max}\limits}
	\newcommand{\argmin}{\mathop{\rm arg~min}\limits}
	\newcommand{\Ker}{{\rm Ker}}
	\newcommand{\rank}{{\rm rank}}
%%%%%%%%%%%%%%%%%%%%%%%%%%%%%%%%%%%%%
	\section{【予備】局所修正法について}
		前回のMTGではそもそも計算の定義が間違っており,実行できませんでした.
		
		今までとそこまで解法の変わり映えしないという理由もあり,修正アルゴリズムを提案したいと考えております.このアルゴリズムでは,SCIで発表したように一度に変形させるのではなく,有限の修正操作を繰り返すことで点を満たすように移動することを考えたいと思います.
		
		それを踏まえて,一度今までの式を振り返ります.
		ある$ \varepsilon_c $および$ \bd{d} $に対して,修正後の長さ分布$ \varepsilon(s) $は,$ \bd{d} $が$ s $によって変化しない仮定の下で次式で表されます.
		\begin{equation}\label{eq:varDiffeq}
			\varepsilon' = -\frac{|u'\zetav_U\times \zetav_L|}{D \cos \alpha} \mathrm{sgn}((\zetav_L \times \zetav_U) \cdot \etav)\varepsilon := \sigma \varepsilon
		\end{equation}
		
		この式が$ s = s_c $で$ \varepsilon=\varepsilon_c $を取るならば,一意に
		\begin{equation}\label{eq:vareq}
			\varepsilon = \varepsilon_c \exp \left[ \int_{s_c}^{s} \sigma ds \right]
		\end{equation}
		と表されます.
		
		$ n $回目の操作の時の空間座標位置は$ \xv_{U,n} = \xv_U + \sum_{j=1}^{n} \varepsilon_i \bd{d}_i$と表され,この時,局所修正を想定するときには下記の要求を満たす必要があります.
		\begin{itemize}
			\item $ \int |\sum_{j=1}^{n} \varepsilon_j \bd{d}_j|^2 ds  $が最小であること
			\item $ \xv_{U,n}(s_c) = \bd{C} $であること
			\item $ \varepsilon $は$ s=s_c $で極値$ \varepsilon_c $をとる.
		\end{itemize}
		これを踏まえて,修正アルゴリズムを以下のように提案します.
		
		順におって説明します.漸化式的に,$i$の情報から$ i+1 $の条件を求めます.
		まずはじめに,$ \sigma_i(s_c)=0 $を確認します.これは定義式から見れる通り,現在のステップ$ \bd{d}_{i+1} $に依存しません.そのため,$ s=s_c $が極値をとる条件であるとは限りません.(少なくとも初期値ではそう)この場合,レトラクションという定義する操作を行います.
		
		レトラクションという操作は次の条件を満たすように曲線を変形させます.はじめに,$ \sigma_i(s_p) =0$を満たす$ s_p $を求めます.次に,$ s=s_p $の時にある$ \varepsilon_{c,p} $だけ動かし,$ s=s_c $の時に極値をとるようにする.つまり,
		\begin{equation}\label{eq:CylinderEq}
			|\tilde{\zetav}_U \times \zetav_L| = 0
		\end{equation}
		これを満たすもののうち,$ \varepsilon_{c,p} \rightarrow \min $を満たすような$ \varepsilon_{c,p},\theta_p,\phi_p $を求めた上で,曲線を変形させます.この操作をレトラクションを定義しています.
		
		次に,$ s=s_c $で極値をとる仮定で話を進めます.この時,$ \phi_{i+1},\theta_{i+1},\varepsilon_{c,i+1} $に従って曲線を変形させる時,局所修正の1番目の条件より,三つの変数は以下を満たします.
		\begin{equation}\label{eq:FirstAmendEq}
			 \int |\sum_{j=1}^{i+1} \varepsilon_j \bd{d}_j|^2 ds
		\end{equation}
		これを$ \bd{\delta}_i =  \sum_{j=1}^{i} \varepsilon_j \bd{d}_j$とおき展開すると次式が成り立つ.
		
		\begin{equation}\label{eq:FirstAmendEqVer2}
			\varepsilon_{c,i+1}^2 \int_{0}^{L} \exp \left(2 \int_{s_c}^s \sigma_i \right) ds + 2 \varepsilon_{c,i+1} \int_{0}^{L} \exp \left( \int_{s_c}^s \sigma_i \right) \bd{d}_{i+1} \cdot \bd{\delta}_i ds + \int_{0}^{L} |\bd{\delta}_i|^2 ds 
		\end{equation}
		これは,シンプルな二次関数と捉えることもできる.つまり,$ \varepsilon_{c,i+1} $が最小となるときは,
		\begin{equation}\label{eq:eps_i+1eq}
			\varepsilon_{c,i+1} = - \frac{\int_{0}^{L} \exp \left( \int_{s_c}^s \sigma_i \right) \bd{d}_{i+1} \cdot \bd{\delta}_i ds}{\int_{0}^{L} \exp \left(2 \int_{s_c}^s \sigma_i \right) ds}
		\end{equation}
		
		
		
	\section{次回のMTGについて(終了後記載)}
	\begin{itemize}
		\item  
		\item 
	\end{itemize}
	###
	\newpage
%\vspace{10cm}
%%%%%%%%%%%%%%%%%%%%%%%%%%%%%%%%%%%%%%
% 3.達成できなかったこととその問題点
	%\articleSPRthree
	
%%%%%%%%%%%%%%%%%%%%%%%%%%%%%%%%%%%%%%

%\vspace{14cm}
%%%%%%%%%%%%%%%%%%%%%%%%%%%%%%%%%%%%%%
	%\articleSPRfour
	%\articleSPRfive
\end{document}
