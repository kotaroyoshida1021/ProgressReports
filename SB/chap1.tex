\section{緒言}
伸縮なしに平面へ展開することのできる可展面は,日常の中で現れる様々な工業製品に現れる.以下にはほんの一例を示す.
■船舶の側面部
船舶における側面部の曲面は,大きな平板を曲げながら溶接することで形成される.こうした形状は,溶接などでの変形は存在し,局所的に可展面制約を守れていない部分も存在するが,全体的に見れば可展面であるとみなすことができる.これらの合板は,Fig~に示すような形によって作成される.
■衣服
衣服はパターンと呼ばれる型紙を設計し,その形通りに布から切り抜き,縫い合わせることで作成する.衣服には大きく分けて2種類あり,Tシャツやスカートなど着装時に比較的「ゆとり」を持つことが求められる製品と,靴や下着類,またはスポーツ品などの着装時に「ゆとり」を持ってはいけないものである.前者の場合に用いられる布地には伸縮性が高いものが用いられる一方,後者の場合には「ゆとり」を持たせないという要求から,伸縮性が高くないものが用いられる.この場合,パターンを縫い合わせることで形成される3次元形状が大きな意味を持つ.
■建築物
建築物においては,

これらの製品の共通点は,その製造過程で平板などを「曲げる」ことによって形状を作る部分にある.可展面はねじれがなく曲げのみによって表現できる曲面であるため,上記のような製品を設計する際には,なるべく可展面に近い形で設計されることが望まれる.