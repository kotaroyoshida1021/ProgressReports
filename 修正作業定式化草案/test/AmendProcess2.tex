\newcommand{\STP}[3]{\det[#1,\; #2,\; #3]}

\section{定式化の流れ}
	\subsection{修正作業の定義}
		本研究が対象する修正作業は,ブラジャーカップにおける下カップの上下接ぎライン形状の変更作業を対象とする.この修正作業には,
		\begin{itemize}
			\item 可展面制約を満たす
			\item 形状が大きく変化しない
			\item 形状がなるべくなめらかである.
		\end{itemize}
		という三つに分けられるのではないかと考える.元の可展面が十分なめらかである場合,この大きく変化しないは,なめらかさの要求を包括すると考えると,この大きく変化しないという目的関数の下,形状を最適化することで,修正作業を定式化することを考える.
		
	\subsection{定式化}
		下ワイヤの弧長パラメータを$ s $とし,上下接ぎラインの弧長パラメータを$ u $とする.上下接ぎラインの空間座標$ \tilde{\bd{x}}_U(s) $が,ある方向ベクトルの分布$ \bd{p}(s) $および$ \varepsilon(s) $に従って$ \tilde{\bd{x}}_U(s) + \varepsilon \bd{p}(s) $という形で変化すると仮定する.この時,分布の接方向ベクトルが以下のように表されるとする.
		\begin{equation}\label{eq:d_eq}
			\bd{d} = \bd{\zeta}_L \cos \phi + \bd{\xi}_L \sin \phi \cos \theta + \bd{\eta}_L \sin \phi \sin \theta
		\end{equation}
		
		これを用いて方向ベクトル$ \bd{p} $を以下のように表す.
		\begin{equation}\label{eq:p_eq}
			\bd{p} = \frac{\int_{0}^{s} \bd{d} ds}{|\int_{0}^{s} \bd{d} ds|}
		\end{equation}
		この曲線が可展開条件を満たすならば,以下の方程式が成立する.
		
		\begin{equation}\label{eq:dev_eq}
			\det [\bd{\zeta}_L,\;\; \bd{d},\;\; \tilde{\bd{x}}_U(s) - \bd{x}_L + \varepsilon \bd{p}(s) ] = 0
		\end{equation}
		この式を$ \varepsilon $について解くと次のように表される.
		\begin{equation}\label{eq:varep_eq}
			\varepsilon = \frac{D\cos \alpha \sin \theta \sin \phi }{\bd{\eta}_L \cdot \bd{p} \sin \theta \cos \phi + \bd{\xi}_L \cdot \bd{p} \sin \theta \sin \theta}
		\end{equation}
		つまり,母線変化を表す空間ベクトルに合わせて,上下接ぎラインの変化量は決定される.
		
		次に,上下接ぎラインの変化量に対する可展面の変化量を定式化する.
		
		変位した可展面の任意のパラメータ$ s,\hat{t} \in [0,L_L] \times [0,D(s)] $における空間座標は,次式のように表される.
		\begin{equation}\label{eq:Sur_eqhat}
			\bd{\hat{x}}_U = \bd{x}_L + \hat{t} \hat{\bd{g}}
		\end{equation}
		ただし,$ \hat{\bd{g}}$は,$ \tilde{\bd{x}}_U + \varepsilon \bd{e}(s) - \bd{x}_L $を自身のノルムで割り,正規化したベクトルを表している.この時,ある$s$に対する曲面の二つの空間座標間の総移動量は,三つの空間座標$ \bd{x}_L,\bd{x}_U,\bd{x}_U + \varepsilon \bd{e} $からなる三角形の面積であると定義する.すなわち,目的関数は次式のように表される.
		\begin{equation}\label{eq:Objective}
			\int_{0}^{L_L} |\varepsilon \bd{e} \times (\bd{x}_U - \bd{x}_L) | ds
		\end{equation}